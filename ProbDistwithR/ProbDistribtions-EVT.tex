
\section{Probability Distributions and EVT}


%-----------------------------------------------------------%
% Part 1
\subsection{Introduction to Probability Distributions}

%-----------------------------------------------------------%
\subsubsection{Random Numbers Generation}
 % Reproducibility

%-----------------------------------------------------------%
% Part 2
\subsection{Advanced Probability Distributions}

%-----------------------------------------------------------%
% Part 3
\subsection{Gambler's Ruin}

%-----------------------------------------------------------%
% Part 4
\subsection{Extreme Value Theory}

%-----------------------------------------------------------%
Extreme Value. The extreme value distribution is often used to model extreme events, such as the size of floods, gust velocities encountered by airplanes, maxima of stock market indices over a given year, etc.; it is also often used in reliability testing, for example in order to represent the distribution of failure times for electric circuits (see Hahn and Shapiro, 1967). The extreme value (Type I) distribution has the probability density function:

\[f(x) = 1/b \times e^{[-(x-a)/b]} \times e^{-e^{[-(x-a)/b]}},    for -8 < x < 8, b > 0\]

where

a	is the location parameter
b	is the scale parameter
e	is the base of the natural logarithm, sometimes called Euler's e (2.71...)
\end{document}
