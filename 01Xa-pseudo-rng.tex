\documentclass[MAIN.tex]{subfiles} 
\begin{document} 


\begin{frame}
	\frametitle{Pseudo-Random Number Generation}
	\textbf{Random Number Generation}
\begin{itemize}
\item To a very high degree computers are deterministic and therefore are not a reliable source of significant amounts of random values. In general pseudo random number generators are used. 
\item The default algorithm in \texttt{R} is Mersenne-Twister but a long list of methods is available. 
\item See the help of \texttt{RNGkind()} to learn about random number generators.
\end{itemize}
\end{frame}	
%======================================================================================= %
\begin{frame}[fragile]
	\frametitle{Pseudo-Random Number Generation}
	\begin{description}		
	\item[\texttt{.Random.seed}] is an integer vector, containing the \textbf{random number generator} (RNG) state for random number generation in \texttt{R}.\\ It can be saved and restored, but should not be altered by the user.
	
	\item[\texttt{RNGkind}] is a more friendly interface to query or set the kind of RNG in use.
	\end{description}
\end{frame}
%======================================================================================= %
\begin{frame}
		\frametitle{Pseudo-Random Number Generation}
		\begin{description}	
\item[\texttt{RNGversion}] can be used to set the random generators as they were in an earlier \texttt{R} version (for reproducibility).
	
\item[\texttt{set.seed}] is the recommended way to specify seeds.
	\end{description}
\end{frame}
%===================================================================================== %
	\begin{frame}[fragile]
	\frametitle{Seed}
		A pseudo random number generator is an algorithm based on a starting point called "\textbf{seed}". \\ If you want to perform an exact replication of your program, you have to specify the seed using the function \texttt{set.seed()}. The argument of \texttt{set.seed()} has to be an integer.
		
		\begin{framed}
		\begin{verbatim}
		> set.seed(1)
		> runif(1)
		[1] 0.2655087
		> set.seed(1)
		> runif(1)
		[1] 0.2655087
		\end{verbatim}
		\end{framed}
	\end{frame}
%======================================================================================= %
\begin{frame}[fragile]
	\frametitle{Pseudo-Random Number Generation}
\textbf{Mersenne Twister}\\
Mersenne Twister(MT) is a pseudorandom number generating algorithm developped by Makoto Matsumoto and Takuji Nishimura (alphabetical order) in 1996/1997. 

\begin{itemize}
	\item It is designed with consideration on the flaws of various existing generators.
	\item The algorithm is coded into a C-source downloadable below.
	\item Far longer period and far higher order of equidistribution than any other implemented generators. 
	% (It is proved that the period is $2^19937-1$, and 623-dimensional equidistribution property is assured.)
\end{itemize}
\end{frame}

%======================================================================================= %
\begin{frame}
	
	\frametitle{Mersenne Twister}
\textbf{Mersenne Twister}\\
\begin{itemize}
\item Fast generation. \\ \textit{(Although it depends on the system, it is reported that MT is sometimes faster than the standard ANSI-C library in a system with pipeline and cache memory.)}
% (Note added in 2004/3: on 1998, usually MT was much faster than rand(), but the algorithm for rand() has been substituted, and now there are no much difference in speed.)
\item Efficient use of the memory. 
%(The implemented C-code mt19937.c consumes only 624 words of working area.)
\end{itemize}	
\end{frame}
%======================================================================================= %
\end{document}	
		