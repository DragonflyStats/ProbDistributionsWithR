\documentclass[MAIN.tex]{subfiles} 
\begin{document} 
%============================================================================ %
\begin{frame}
	\frametitle{Benford's Law}
\begin{itemize}
\item Benford's law, also called the \textbf{First-Digit Law}, refers to the frequency distribution of digits in many (but not all) real-life sources of data. 
\item In this distribution, 1 occurs as the leading digit about 30\% of the time, while larger digits occur in that position less frequently: 9 as the first digit less than 5\% of the time. 
\item Benford's law also concerns the expected distribution for digits beyond the first, which approach a uniform distribution.
\end{itemize}
\end{frame}
%============================================================================ %
\begin{frame}
	\frametitle{Benford's Law}
\begin{itemize}
\item It has been shown that this result applies to a wide variety of data sets, including electricity bills, street addresses, stock prices, population numbers, death rates, lengths of rivers, physical and mathematical constants, and processes described by power laws (which are very common in nature). \item It tends to be most accurate when values are distributed across multiple orders of magnitude.
\end{itemize}
\end{frame}
%============================================================================ %
\begin{frame}
\frametitle{Benford's Law}
\large
\begin{itemize}
\item The graph here shows Benford's law for base 10. 
\item There is a generalization of the law to numbers expressed in other bases (for example, base 16), and also a generalization from leading 1 digit to leading n digits.
\item It is named after physicist Frank Benford, who stated it in 1938, although it had been previously stated by Simon Newcomb in 1881.
\end{itemize}
\end{frame}
%============================================================================ %
\begin{frame}[fragile]
\frametitle{Benford Distribution}
%It is due to Benford 1938[1] and Newcomb 1881[2].
\begin{framed}
\begin{verbatim}
> # The Benford Distribution is the distribution of 
> # the first digit of a number. 
>
> library(VGAM)
>
> dbenf(c(1:9))
[1] 0.30103000 0.17609126 0.12493874 
[4] 0.09691001 0.07918125 0.06694679
[7] 0.05799195 0.05115252 0.04575749
\end{verbatim}
\end{framed}
\end{frame}
\end{document}