\newpage

\subsection{The Exponential Distribution}


The exponential distribution describes the arrival time of a randomly recurring independent event sequence. If $\mu$ is the mean ``duration" or ``waiting time" for the next event recurrence, its probability density function is given as

\[f(x;\lambda) = \begin{cases}
\lambda e^{-\lambda x}, & x \ge 0, \\
0, & x < 0.
\end{cases}\]
The parameter $\lambda$  is called \textbf{\emph{rate}} parameter. It is the inverse of the expected duration ($\mu$).\\
\[\mu= \frac{1}{\lambda}\].The cumulative distribution function of an exponential distribution is

\[
F(x;\lambda) = \begin{cases}
1-e^{-\lambda x}, & x \ge 0, \\
0, & x < 0.
\end{cases}\]

If the expected duration is 5 ( e.g. five minutes) then the rate parameter value is 0.2. When implemented with \texttt{R}, the exponential distribution is characterized by the single parameter \texttt{\textbf{rate}}.The commands follow the same kind of naming convention, and the names of the commands are \texttt{\textbf{dexp()}}, \texttt{\textbf{pexp()}}, \texttt{\textbf{qexp()}}, and \texttt{\textbf{rexp()}}. A few examples are given below to show how to use the various commands. 

\subsubsection{Exponential Distribution : Problem}
Suppose the mean checkout time of a supermarket cashier is three minutes. Find the probability of a customer checkout being completed by the cashier in less than two minutes, three minutes and four minutes. 
(i.e. what percentage of ``waiting times" are less than two, three and four minutes?)

\textbf{\emph{Solution}} The checkout processing rate is equals to one divided by the mean checkout completion time. 
Hence the processing rate is 1/3 checkouts per minute. We then apply the function \texttt{\textbf{pexp()}} of the exponential distribution with rate=1/3. 
Also compute the probability that the waiting time exceeds 5 minutes.


\begin{verbatim}
> pexp(2,rate=1/3)
[1] 0.4865829
> pexp(3,rate=1/3)
[1] 0.6321206
> pexp(4,rate=1/3)
[1] 0.7364029
> pexp(5,rate=1/3,lower=FALSE)
[1] 0.1888756
\end{verbatim}
The probabilities of finishing a checkout in under two, three and four minutes by the cashier are approximately 48.6 \%, 63.21\% and 73.6\% respectively. 

What is the median waiting time? To answer this question we would use the \texttt{\textbf{qexp()}} function. Recall that the median is value of $x$ such that $P(X \leq x) = 0.50$.
Also determine the first and third quartiles $Q_1$ and $Q_3$.
\begin{verbatim}
> qexp(0.5,rate=1/3)
[1] 2.079442
> qexp(0.25,rate=1/3)
[1] 0.8630462
> qexp(0.75,rate=1/3)
[1] 4.158883
\end{verbatim}
\newpage
