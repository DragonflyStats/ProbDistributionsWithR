\documentclass[12pt, a4paper]{article}
\usepackage{epsfig}
\usepackage{subfigure}
%\usepackage{amscd}
\usepackage{amssymb}
\usepackage{amsbsy}
\usepackage{amsthm, amsmath}
%\usepackage[dvips]{graphicx}
\usepackage{natbib}
\bibliographystyle{chicago}
\usepackage{vmargin}
% left top textwidth textheight headheight
% headsep footheight footskip
\setmargins{3.0cm}{2.5cm}{15.5 cm}{22cm}{0.5cm}{0cm}{1cm}{1cm}
\renewcommand{\baselinestretch}{1.2}
\pagenumbering{arabic}
\theoremstyle{plain}
\newtheorem{theorem}{Theorem}[section]
\newtheorem{corollary}[theorem]{Corollary}
\newtheorem{ill}[theorem]{Example}
\newtheorem{lemma}[theorem]{Lemma}
\newtheorem{proposition}[theorem]{Proposition}
\newtheorem{conjecture}[theorem]{Conjecture}
\newtheorem{axiom}{Axiom}
\theoremstyle{definition}
\newtheorem{definition}{Definition}[section]
\newtheorem{notation}{Notation}
\theoremstyle{remark}
\newtheorem{remark}{Remark}[section]
\newtheorem{example}{Example}[section]
\renewcommand{\thenotation}{}
\renewcommand{\thetable}{\thesection.\arabic{table}}
\renewcommand{\thefigure}{\thesection.\arabic{figure}}
\title{Research notes: Extreme Value Theory}
\author{ } \date{ }


\begin{document}
\author{Kevin O'Brien}
\title{R for Extreme Value Theory}

% http://www.floodrisk.org.uk/index.php?option=com_content&view=article&id=146&catid=0&Itemid=41




\newpage
\section{using  \texttt{R}  to study Extreme Value Theory}
The statistical programming environment \texttt{R} supports many of the most frequently used EVT analyses.

\subsection{ Probability distributions supported by \texttt{R}}

\begin{itemize}
\item The Gumbell dsitribution
\item Generalized pareto distribution (GPD)
\end{itemize}





%----------------------------------------------------------------------------------------%
\section{The GWSDAT \texttt{R} package}
%--http://meetingorganizer.copernicus.org/EGU2011/EGU2011-8678.pdf
The GroundWater Spatio-Temporal Data Analysis Tool, or GWSDAT, has been
developed by Shell Global Solutions to facilitate the analysis and reporting of trends in groundwater monitoring
data. GWSDAT automatically uploads data from MS Excel to generate a user- friendly interface, which allows
users to scroll through the groundwater monitoring history of a site, exporting trend plots and graphics as required.
The underlying statistical calculations and graphical output are generated using the open- source statistical
program  \texttt{R} (www.r-project.org)
\section{Analyzing Geospatial data}
Methods for visualising and handling Spatial data include the \texttt{R} packages \textbf{\emph{deldir}},  \textbf{\emph{sp}},  \textbf{\emph{splancs}} and \textbf{\emph{ maptools}}.

%----------------------------------------------------------------------------------------%


\begin{itemize}

\item \textbf{\emph{RHydro}}
 is an experimental package providing dedicated classes and methods for hydrological modelling and data analysis.

\item \textbf{\emph{hydroTSM}} - contains functions for
management, analysis, interpolation and plotting of
 time series used in hydrology and related environmental sciences.
 In particular, this package is highly oriented to hydrological modelling tasks.
 The focus of this package has been put in providing a collection of tools useful for the daily work of hydrologists.


\item \textbf{\emph{sp}} - A package that provides classes and
methods for spatial data. The classes document where the spatial
location information resides, for 2D or 3D data. Utility functions
are provided, e.g. for plotting data as maps, spatial selection,
as well as methods for retrieving coordinates, for subsetting,
print, summary, etc.

\item \textbf{\emph{maptools}} - Set of tools for manipulating and
reading geographic data, in particular ESRI shapefiles; C code
used from shapelib. It includes binary access to GSHHS shoreline
files. The package also provides interface wrappers for exchanging
spatial objects with packages such as PBSmapping, spatstat, maps,
RArcInfo, Stata tmap, WinBUGS, Mondrian, and others.

\item \textbf{\emph{evd}} - this package extends simulation,
distribution, quantile and density functions to univariate and
multivariate parametric extreme value distributions, and provides
fitting functions which calculate maximum likelihood estimates for
univariate and bivariate maxima models, and for univariate and
bivariate threshold models.



\end{itemize}



%\chapter{Geospatial data}
\section{Contour Lines}
A contour line (also isoline or isarithm) of a function of two variables is a curve along which the function has a constant value.
In cartography, a contour line (often just called a "contour") joins points of equal elevation (height) above a given level, such as mean sea level.
 A contour map is a map illustrated with contour lines, for example a topographic map, which thus shows valleys and hills, and the steepness of slopes.
The contour interval of a contour map is the difference in elevation between successive contour lines.

\end{document}
