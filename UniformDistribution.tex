\subsection{The Uniform Distribution}

A random variable X is called a continuous uniform random variable over the interval $(a,b)$ if it's probability density function is given by
\[ f_{X}(x) = { 1 \over b-a} \hspace{1cm} \mbox{ when } a \leq x \leq b \mbox{     (otherwise } f_X(x) = 0 ) \]
The corresponding cumulative density function is
\[ F_x(x) = { x-a \over b-a} \hspace{1cm} \mbox{ when } a \leq x \leq b\]


\begin{center}
\includegraphics[scale=0.35]{6AUniform}
\end{center}
The continuous distribution is very simple to understand and implement, and is commonly used in computer applications (e.g. computer simulation).
It is known as the `Rectangle Distribution' for obvious reasons. We specify the word ``continuous" so as to distinguish it from it's discrete equivalent: the discrete uniform distribution.



The continuous uniform distribution is characterized by the following parameters


\begin{itemize}
\item The lower limit $a$ (or with \texttt{R}: \texttt{\textbf{min}} )

\item The upper limit $b$ (or with \texttt{R}: \texttt{\textbf{max}} )

\item We denote a uniform random variable $X$ as $X \sim U(a,b)$
\end{itemize}
It is not possible to have an outcome that is lower than $a$ or larger than $b$ ,i.e. $ P(X < a) = P(X > b) = 0$.

We wish to compute the probability of an outcome being within a range of values.We shall call this lower bound of this range $L$ and the upper bound $ U$. Necessarily $L$ and $U$ must be possible outcomes.The probability of $X$ being between $L$ and $U$ is denoted $P( L \leq X \leq U )$.


In the absence of the specified upper and lower limits, the default values of 0 and 1 would be used.

\subsubsection{Generating Random Values}
The most common use of the uniform distribution is the generation of uniformly distributed random variables. 


The relevant command is \texttt{\textbf{runif()}}. 
Functions that can be used to manage precision are useful when generating random values.

\begin{verbatim}
> runif(10)
 [1] 0.2558100 0.8738507 0.4521578 0.7868320 0.2310644 0.5265236
 [7] 0.9041761 0.3948904 0.1928505 0.5793142
> runif(10,min=1,max=7)
 [1] 2.822485 4.603547 5.794749 3.766398 2.016349 2.116504 5.863682
 [8] 3.911420 3.434373 6.986899
>
> Another Way of Simulating Dice Rolls
> floor(runif(10,min=1,max=7))
 [1] 1 5 1 6 6 2 1 2 6 3
\end{verbatim}


