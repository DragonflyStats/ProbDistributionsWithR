Extreme value theory is a branch of statistics dealing with the extreme deviations from the median of probability distributions. The general theory sets out to assess the type of probability distributions generated by processes. Extreme value theory is important for assessing risk for highly unusual events, such as 100-year floods.

%=================================================================================%

\subsection*{Extreme Value Distributions}
Extreme Value Theory (EVT), which as the name suggests is useful for modelling 
extreme events, has become more popular amongst actuaries in recent years. For 
example EVT might be useful when calculating an insurer’s ICA, which by definition 
considers events at the 1 in 200 year level. It may also be helpful for pricing a high 
XL reinsurance layer. At the very least, EVT simply provides the actuary with an 
alternative set of distributions with which to model losses. 

Two distributions for EVT are:
\begn{itemize}
\item Generalised Extreme Value (GEV) distribution. This family describes the 
distribution of the maxima of sets of observations, for example the largest annual 
claim on an insurance contract. It encompasses three classes of distribution: 
Gumbel (shape parameter, ξ = 0), Frechet (ξ > 0) and Weibull (ξ < 0). 

\item Generalised Pareto Distribution (GPD). This distribution describes exceedences 
over a threshold and is perhaps more convenient for modelling insurance claims.
Functions for fitting both distributions, including diagnostic functions are available in 
add on packages for R. We shall use functions from two such packages in this 
example. We shall consider a simple example, from “The Modelling of Extremal 
Events” by D Sanders (2005), which looks to fit a GPD.
\end{itemize}

We start by loading up the data and viewing a histogram of the data

\section*{Extreme Value Theory}
\subsection*{Extreme Order Statistics}
Let $G_k$ be the distribution function of $X_{(k)}$
\[ G_1 = 1 - [1-F_{(1)}]^n \]

n length of set of bivariate n=2

\[ G_2 = F_{(1)}^2 \]


%--------------------------------------------------------------------%

\subsection*{Conditional Tail Expectation}

Exceedance : Event that an observation from population of X exceeds a given threshold c
\[ P (X > C) \]



\subsection*{Approaches}

Two approaches exist today:

The difference between the two theorems is due to the nature of the data generation. 

For Theorem I the data are generated in full range, while in Theorem II data is only generated when it surpasses a certain threshold, called Peak Over Threshold models (POT). The POT approach has been developed largely in the insurance business, where only losses (pay outs) above a certain threshold are accessible to the insurance company. Strangely, this approach is often used for cases where Theorem I applies, which creates problems with the basic model assumptions.

Extreme value distributions are the limiting distributions for the minimum or the maximum of a very large collection of independent random variables from the same arbitrary distribution. Emil Julius Gumbel (1958) showed that for any well-behaved initial distribution (i.e., F(x) is continuous and has an inverse), only a few models are needed, depending on whether you are interested in the maximum or the minimum, and also if the observations are bounded above or below.

\subsection*{Applications}

Applications of extreme value theory include predicting the probability distribution of:
\begin{itemize}
\item Extreme floods
\item The amounts of large insurance losses
\item Equity risks
\item Day to day market risk
\item The size of freak waves
\item Mutational events during evolution
\item Large wildfires[1]
\end{itemize}
It can be applied to some characterization of the distribution of the maxima of incomes, like in some surveys done in virtually all the National Offices of Statistics
Estimate fastest time humans are capable of running the 100-meter sprint.[2]
Pipeline failures due to pitting corrosion
