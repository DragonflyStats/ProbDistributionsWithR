\documentclass[]{article}
\usepackage{framed}
\begin{document}
\tableofcontents

\section{Analysis of extreme values using \texttt{R} and \texttt{extRemes} }

\subsection{Contributed packages}
\begin{framed}
\begin{verbatim}

# Install some useful packages. Need only do once.
install.packages( c("fields", # A spatial stats package.
"evd", # An EVA package.
"evdbayes", # Bayesian EVA package.
"ismev", # Another EVA package.
"maps", # For adding maps to plots.
"SpatialExtremes")

# Now load them into R. Must do for each new session.
library( fields)
library( evd)
library( evdbayes)
library( ismev)
library( SpatialExtremes)
\end{verbatim}
\end{framed}

\subsection{Background on Extreme Value Analysis (EVA)}
%Motivation
\textbf{Sums, averages and proportions (Normality)}
\begin{itemize}
\item Central Limit Theorem (CLT)
\item Limiting distribution of binomial distribution
\end{itemize}
\textbf{Extremes}
\begin{itemize}
\item Normal distribution inappropriate
\item Bulk of data may be misleading
\item Extremes are often rare, so often not enough data
\end{itemize}

\newpage
\section{Generalized Extreme Value (GEV) distribution}
The generalized extreme value (GEV) distribution is a family of continuous probability distributions developed within extreme value theory to combine the Gumbel, Fréchet and Weibull families also known as type I, II and III extreme value distributions. 

By the extreme value theorem the GEV distribution is the only possible limit distribution of properly normalized maxima of a sequence of independent and identically distributed random variables. 

\begin{itemize}
\item Gumbel
\item Fréchet
\item Weibull
\end{itemize}
The Gumbel distribution is used to model the distribution of the maximum (or the minimum) of a number of samples of various distributions. Such a distribution might be used to represent the distribution of the maximum level of a river in a particular year if there was a list of maximum values for the past ten years. It is useful in predicting the chance that an extreme earthquake, flood or other natural disaster will occur.
 


Distribution fitting with confidence band of a cumulative Gumbel distribution to maximum one-day October rainfalls.
Gumbel has shown that the maximum value (or last order statistic) in a sample of a random variable following an exponential distribution approaches the Gumbel distribution closer with increasing sample size.[4]
In hydrology, therefore, the Gumbel distribution is used to analyze such variables as monthly and annual maximum values of daily rainfall and river discharge volumes,[5] and also to describe droughts.[6]

Gumbel has also shown that the estimator r / (n+1) for the probability of an event - where r is the rank number of the observed value in the data series and n is the total number of observations - is an unbiased estimator of the cumulative probability around the mode of the distribution. Therefore, this estimator is often used as a plotting position.

The blue picture illustrates an example of fitting the Gumbel distribution to ranked maximum one-day October rainfalls showing also the 90\% confidence band based on the binomial distribution. The rainfall data are represented by the plotting position r / (n+1) as part of the cumulative frequency analysis.

In number theory, the Gumbel distribution approximates the number of terms in a partition of an integer[7] as well as the trend-adjusted sizes of record prime gaps and record gaps between prime constellations.[8]

 
\textbf{Light tail}
\begin{itemize}
\item Domain of attraction for many common distributions
(e.g., normal, lognormal, exponential, gamma)
\end{itemize}

\end{document}
